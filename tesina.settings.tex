%!TEX spellcheck = es-AR
% !TEX encoding = UTF-8 Unicode
\documentclass[spanish,12pt,a4paper,openright]{memoir}
\usepackage[spanish,es-tabla]{babel}
\usepackage[utf8]{inputenc}
\usepackage{amsmath,amssymb,latexsym}
% Doe's not play well with memoir
%\usepackage{subfigure}
\usepackage{pifont}
\usepackage[linktocpage=true]{hyperref}
\usepackage{MnSymbol}
\usepackage{algorithm}
\usepackage[noend]{algpseudocode}
\usepackage{wrapfig}
\usepackage{xspace}
\usepackage{multirow}
\usepackage{adjustbox}
\usepackage[figuresleft]{rotating}
\usepackage{rotfloat}
\usepackage[final]{fixme}
\usepackage{multicol}
\usepackage{tabularx,ragged2e,booktabs,caption}
\usepackage{todonotes}
\usepackage{graphicx}
\usepackage{tikz}
\newsubfloat{figure}% Allow subfloats in figure environment
\usetikzlibrary{automata,petri,positioning,shapes,snakes,arrows,backgrounds,babel}
\tikzstyle{tplace}=[circle,draw,inner sep=1.8mm]
\usepackage{amsthm}
% Encabezado "lindo"
% Memoir y fancyhdr definen ambos esto. Pero lo definen igual, así que no hay problema
% Fuente: http://tex.stackexchange.com/questions/37868/fancyhdr-and-memoir
\let\footruleskip\undefined
\usepackage{fancyhdr}
% Saca los espacios adicionales de las listas tipo itemize
\usepackage{enumitem}
\usepackage{hypcap}
\setlist{nolistsep}
% Para armar código bonito
\usepackage{minted}
\usemintedstyle{bw}
\usepackage{xcolor}
\usepackage[framemethod=default]{mdframed}
\definecolor{bg}{rgb}{0.95,0.95,0.95}
\mdfdefinestyle{codebox}{backgroundcolor=bg,skipabove=10,linewidth=0}
\usepackage[most]{tcolorbox}

\definecolor{mygreen}{rgb}{0,0.5,0}

\input{macros}
\newtheorem{theorem}{\normalfont \textbf{Teorema}}
\newtheorem{definition}{\normalfont \textbf{Definición}}[section]
\newtheorem{assumption}{\normalfont \textbf{Assumption}}
\newtheorem{proposition}{\normalfont \textbf{Proposition}}
\newtheorem{corollary}{\normalfont \textbf{Coroloario}}
\newtheorem{example}{\normalfont Ejemplo}[section]
\newtheorem{remark}{\normalfont\textbf{Observación}}

\def\chapterautorefname{Capítulo}

%% Nada anda :(
% NO QUIERO QUE DIGA SUBSUBSUB sección!
\def\sectionautorefname{Sección}
\def\subsectionautorefname{Sección}
\renewcommand{\subsubsectionautorefname}{Sección}
\let\subsectionautorefname\sectionautorefname
\let\subsubsectionautorefname\sectionautorefname



\def\exampleautorefname{Ejemplo}
\def\definitionautorefname{Definición}
\def\propositionautorefname{Prop.}
\def\lemmaautorefname{Lema}
\def\remarkautorefname{Observación}
\def\assumptionautorefname{Assumption}
\def\theoremautorefname{Teorema}
\def\algorithmautorefname{Algoritmo}
\def\tableautorefname{Tabla}
\def\figureautorefname{Figura}
\def\subfigureautorefname{Figura}
\def\equationautorefname{Ecuación}

%% Activa la numeracion de subsecciones
%\setchaptersnumdepth{subsection}
%%%\setcounter{secnumdepth}{3} 
% Mostrar sub secciones en índice
\setcounter{tocdepth}{3}

\hypersetup{pdftex,colorlinks=true,allcolors=blue}


\pagestyle{fancy}
\fancyhf{}
\fancyhead[EL]{\nouppercase\leftmark}
\fancyhead[OR]{\nouppercase\rightmark}
\fancyhead[ER,OL]{\thepage}



