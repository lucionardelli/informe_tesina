% Capítulo 2 - Nociones preliminares
\chapter[Nociones preliminares]{Nociones preliminares}
\label{chap:2}

En este capítulo introduciremos las definiciones y conceptos básicos con los que se trabajará durante el desarrollo de esta tesina. Se presentarán las \textit{redes de Petri} (\autoref{sec:2.petrinets}), la \textit{minería de procesos} (\autoref{sec:2.process mining}), \textit{Parikh Vectors} (\autoref{sec:2.parikh}),
\textit{Domínios numéricos abstractos} y su aplicación al desubrimiento de procesos (\autoref{sec:2.discovery}) así como los conceptos utilizados de \textit{logs} (\autoref{sec:2.logs}) e \textit{información negativa} (\autoref{sec:2.negative}).
\\

Por último, utilizando estos conceptos, plantearemos de manera concreta el problema a resolver en los capítulos siguientes (\autoref{chap:2.problem}).

\section{Redes de Petri}
\label{sec:2.petrinets}
Las \textbf{redes de Petri}, introducidas en el año 1962 por el matemático
Carl Adam Petri, consisten en una generalización de la teoría de automátas.
Nacidas de la necesida de de representar sistemas dinámicos con eventos concurrentes;
forman un lenguaje gráfico y matemático con una semántica formal.
Desde su creación, han sido utilizadas para, entre otros usos, para representación,
análisis, verificación y simulación de sistemas a eventos discretos con comportamiento dinámico\cite{Murata89}.

Una red de Petri esta formada por dos componentes, la red propiamente dicha y un conjunto
de \dquote{fichas} asignados a ciertos nodos de la misma.

La primera de estas componentes consiste en un grafo bipartito dirigido ponderado,
cuyos nodos se separan en los conjuntos dijuntos 
llamados lugares -\textit{places}- y transiciones -\textit{transitions}-.
Los arcos del grafo se dirigen tanto de los places a las transitions como viceversa.

En el momento de graficar una red de Petri, los places, por los general, se
representan mediante círculos y las transiciones mediante cajas o, simplemente, barras.

La segunda componente de una red de Petri consiste en la asignación
de un número entero de fichas o marcas a cada uno de los places, 
utilizado para simular el comportamiento dinámico y concurrente del sistema.
A la distribución de estas marcas sobre cada uno de los places, se la denomina
\textit{marking} y corresponde al estado en el cual se encuentra en cada momento el sistema.
La distribución inicial de estas fichas, por tanto, se conoce como \textit{marking inicial}.

Gráficamente, esta asignación se representa de manera distribuída, indicando dentro 
de cada place el número entero asignado por el marking, o bien, dibujando una 
cantidad de puntos igual a dicho número.
\\

Formalmente, una red de Petri es una 4-upla $(P,T,F,M_0)$ donde $P$ y $T$\footnotemark[1]
representan los conjuntos finitos y disjuntos de places y transitions respectívamente.
Por su parte, las transiciones ponderadas vienen dadas por \mbox{$F:(P \times T) \cup (T \times P)  \to \nat$}
y un marking $M$ viene dado por la función \mbox{$M:P \to \nat$}.
En particular, llamamos $M_0$ al estado inicial de la red, i.e. el marking inicial.

\footnotetext[1]{ En este trabajo, utilizamos el mismo símbolo $T$ para denotar el conjunto de transiciones -\textit{transitions}-
    de una red de Petri así como para el alfabeto de eventos de las trazas de un log.
    
    Esta colisión es intencionada ya que, en nuestro modelo, cada transición se corresponde biunivocamente
    con una actividad del log (solo consideramos la actividad completa y no el inicio/fin como en otros enfoques).
    Por esta razón, las transiciones silenciosas (aquellas donde su ejecución no puede ser observada)
    no son admitidas en la red y por la cual dos transiciones diferentes no pueden representar una misma acción.
}

Denotamos el \textit{preset} y \textit{postset} de un place $p$ como $\preset{p}$ y $\postset{p}$ respectivamente,
los cuales se definen formalmente como: $\preset{p} =  \{t \in T ~|~ F(t,p) > 0 \}$
y $\postset{p} = \{t \in T ~|~ F(p,t) > 0 \}$.

Se le llama red de Petri \emph{pura} a una red que no posea ciclos de longitud uno, i.e.
$\forall p \in P:~ {\preset{p}} \mathrel{\cap} {\postset{p}} = \emptyset$.
En adelante, asumiremos que todas las redes de Petri con las que trabajamos son puras.
Esto es una consecuencia de utilizar la teoría de poliedros. Es importante destacar que esto no es una 
restricción importante ya que podemos, de manera sistemática, relajar un modelo no puro agregando
places ficticios y convertir cualquier lazo en un ciclo de longitud dos para obtener una red de Petri pura.
\\

Como notamos anteriormente, las redes de Petri se utilizan para representar sistemas 
dinámicos; este comportamiento dinámico está definido mediante las \emph{reglas de evolución}.

Decimos que una cierta transición $t \in T$ esta \emph{habilitada},
en un cierto marking $M$ si \mbox{$\forall p \in P, M(p) \ge F(p,t) $}.

Ejecutar una transición $t$ en un marking induce un nuevo marking $M'$
definido de manera incremental, como  \mbox{$M'(p) = M(p) - F(p,t) +  F(t,p)$},
para cualquier $p \in P$. A esta evolución la notamos \firing{M}{t}{M'}.

Una cierta secuencia de transiciones \mbox{$\sigma = t_1,t_2, t_3, \ldots, t_n$}
se dice ejecutable si existe una secuencia de markings \mbox{$\omega = M_1, M_2, \ldots, M_n$}
tal que ocurra la evolución
\firing {\firing {\firing {\firing {M_0} {t_1} {M_1}} {t_2} {M_2} } {t_3} {\cdots}} {t_n} {M_n}.

Dada una red de Petri $N$, llamamos $\Language(N)$ al lenguaje de la misma, i.e.
el conjunto de secuencias de transitions ejecutables sobre $N$.
Por su parte, al conjunto de markings alcanzable partiendo desde el marking inicial $M_0$,
llamado \emph{conjunto alcanzable} de $N$, lo notamos $\rs(N)$.

\begin{figure}[t]
  	\centering
    \input{img/ineq_net1}
    \caption{Una red de Petri (pura) con pesos no unitarios.}
    \label{fig:pn1}
\end{figure}

Consideremos ahora un place $p$ con
\mbox{$\preset{p}=\{x_1,\ldots,x_k\}$},
\mbox{$\postset{p}=\{y_1,\ldots,y_l\}$} y una función de transición 
$F$ igual a la función constante 1.
Siendo $M_0(p)$ la cantidad de tokens en el marking inicial 
para el place $p$, entonces, la siguiente ecuación
se satisface para cualquier secuencia de eventos $\sigma$

\beql{1place_cte}
M(p) = M_0(p) + \widehat\sigma(x_1)+\cdots +\widehat\sigma(x_k) -
\widehat\sigma(y_1)-\cdots -\widehat\sigma(y_l).
\eeq

La ecuación \eqref{eq:1place_cte}, puede generalizarse para permitir arcos
ponderados como

\beql{1place_wei}
M(p) = M_0(p) + \sum_{x_i \in \preset{p}}F(x_i,p)\cdot
\widehat\sigma(x_i) -
\sum_{y_i\in\postset{p}}F(p,y_i)\cdot\widehat\sigma(y_i).
\eeq

Si extendemos \eqref{eq:1place_wei} a todos los places de una red de Petri,
utilizando la notación matricial, tenemos

\beql{matrix_eq}
M = M_0 + A \cdot \widehat\sigma 
\eeq

donde $M$ y $M_0$ son vectores y $A$ representa la \emph{matriz de incidencia}
del grafo; $A$ posee $|P|$ filas y $|T|$ columnas y representa las conexiones 
existentes en la red.
A la ecuación \eqref{eq:matrix_eq}, se la llama \emph{ecuación de marking} de una red
de Petri~\cite{Murata89}.

Ademas, llamamos \emph{conjunto potencialmente alcanzable}
al conjunto de soluciones de la siguiente ecuación

\beql{matrix_ineq}
M = M_0 + A \cdot \widehat\sigma \geq 0
\eeq

y lo notamos $\prs(N)$.
Utilizando la ecuación \eqref{eq:matrix_ineq} resulta simple definir el concepto
de complejidad de una red de Petri que utilizamos. El mismo consiste en la suma
de los valores absolutos de todos los coeficientes de la matriz $A$ con la suma
de los valores absolutos del vector de $M_0$.

A continuacion, a modo de ejemplo, construímos la ecuación \eqref{eq:matrix_ineq} de
la red de Petri de la ~\autoref{fig:pn1}, la cual es

\beq
 \left[\begin{array}{c} 1 \\ 6 \end{array} \right] +
\left[\begin{array}{rr} 1 & -1 \\ -2 & 3 \end{array} \right]
\cdot
\left[\begin{array}{c} \widehat\sigma(x) \\ \widehat\sigma(y) \end{array}
\right]
\geq \left[\begin{array}{c} 0 \\ 0 \end{array} \right]
\eeq

Es importante ver que todo marking alcanzable de una red de Petri
satisface \eqref{eq:matrix_ineq}, sin embargo, lo opuesto no siempre es cierto.
En general, pueden existir markings no alcanzables para
los cuales \eqref{eq:matrix_ineq} se satisface, i.e. $\rs(N) \subseteq \prs(N)$.

\section{Minería de procesos} 
\label{sec:2.process mining}

\subsection{Logs} 
\label{sec:2.logs}

La minería de procesos tiene su punto inicial en la recopilación de una serie de \textit{log de eventos}. 
Un evento refiere a una actividad, observable o no, de un proceso y
un log de enventos -o simplemente \textit{log}- se define como una secuencia
ordenada de estos eventos, por lo general, generadas de manera automática.
Un log corresponde a una instancia única de la ejecución de un sistema, llamada, instancia de proceso o caso y 
durante la minería de procesos, suelen utilizarse varíos casos de sobre un mismo proceso.

Un log posee los siguientes elementos constitutivos fundamentales: un conjunto de actividades,
un cierto orden temporal en el cual ocurren y un caso o instancia de proceso que agrupe dichas actividades.
Sin embargo, un log pueden contener, además, datos adicionales que, aunque no sean fundamentales, pueden
ayudar a la minería de procesos otorgándole mayor nivel de detalle y claridad.

La tabla \autoref{tab:log_ex} muestra un ejemplo parcial de un conjunto de logs con información adicional (i.e. medio, actividad, nivel) que permite un entendimiento mayor del devenir del proceso,

\begin{table}[t]
\tiny

\def\sep{\hspace{10pt}}
\begin{tabular}{r   c   l   l   r   c   c}
  \tiny \#Caso
& \tiny Timestamp
& \tiny Medio
& \tiny Tipo
& \tiny Urgencia
& \tiny Usr
& \tiny Grupo
\\
\midrule
 970 & 2015-12-31 23:55:12 & Teléfono & Aviso & 0 & Andrés & Nivel 1\newrow
 970 & 2015-12-31 23:56:22 & Mail & Reclamo & 0 & Andrés & Nivel 1\newrow
 970 & 2016-01-01 00:05:11 & Teléfono & Reclamo & 1 & Andrés & Nivel 1\newrow
 971 & 2016-01-01 00:30:44 & Mail & Abierto & 2 & Gonzalo & Nivel 1\newrow
 971 & 2016-01-01 00:30:44 & Mail & Resuelto & 2 & Gonzalo & Nivel 1\newrow
 970 & 2016-01-01 00:30:44 & Teléfono & Derivado & 0 & Andrés & Nivel 2\newrow
 312 & 2016-01-01 15:30:01 & Mail & En progr. & 3 & Gonzalo & Nivel 3\newrow
 42 & 2016-05-04 00:00:01 & Mail & Resuelto & 10 & Lucio & Nivel x\newrow
 42 & 2016-05-04 00:00:01 & Mail & Resuelto & 10 & Lucio & Nivel x\newrow
 $\vdots$ & $\vdots$ & $\vdots$ & $\vdots$ & $\vdots$ & $\vdots$ & $\vdots$ \newrow
\end{tabular}
\vspace{0pt}
\caption{\tiny Un ejemplo artificial de un log de eventos con información adicional.}
\label{tab:log_ex}
\end{table} 


Formalmente, dado un alfabeto $T$\footnote{Ver nota previa referente a la colisión de nombres} representando
al conjunto de posibles eventos, un log se define como un conjunto $\pmlog \in \mathcal{P}(\eventstar)$\footnote{
Los logs pueden ser definidos de manera más general como multiconjuntos donde algunos comportamientos
pueden ser observados múltiples veces. No consideramos este enfoque ya que no consideramos 
la frecuencia de cada traza; las redes solo contemplan la presencia o ausencia de un cierto comportamiento.}. 


Los elementos del log \pmlog, llamados \textit{trazas}, son secuencias de eventos de $T$,
i.e. $\sigma \in \eventstar$.
Dados un numero natural $1 \leq k \leq n$ y una traza $\sigma=\sigma_1,\sigma_2,\dots,\sigma_n$, a la traza
$\sigma_1,\sigma_2,\dots,\sigma_k$ se la llama prefijo -\textit{prefix}- de $\sigma$.

Abusando de notación,  si una cierta traza $\sigma'$ es el prefijo de alguna traza de $\pmlog$, se dice que 
$\sigma' \in \pmlog$ .

% The problem of process discovery requires the computation of a model $\model$ that adequately represents
% a log $\pmlog$.
% A model $\model$ is {\em overfitting} with respect to log $\pmlog$ if it is too specific and too much driven
% by the information in $\pmlog$. On the other hand, $\model$ is an {\em underfitting} model for $\pmlog$ if the behavior
% of $\model$ is too general and allows for things ``not supported by evidence'' in $\pmlog$.  Whereas overfitting
% denotes lack of generalization, underfitting represents too much generalization. A good balance between
% overfitting and underfitting is a desired feature in any process discovery algorithm~\cite{AalstBook}. There exists several formalism for modeling
% processes; those include transition systems, workflow nets, BPMN and causal nets between others. In this article we will discover processes modeled
% as Petri nets.

\subsection{Minería de procesos} 
\label{sec:2.process mining subsection}

Process discovery techniques aim
at extracting from a log $\pmlog$ a process model $N$ (e.g., a Petri net) with the goal to elicit the process
underlying in ${\sys}$. By relating the behaviors of $\pmlog$, $\Language(N)$ and ${\sys}$,
particular concepts can be defined~\cite{BuijsDA14}. A log is \emph{incomplete} if ${\sys} \setminus \pmlog \ne
\emptyset$. A model $N$ \emph{fits} log $\pmlog$ if $\pmlog \subseteq \Language(N)$. A model is
\emph{precise} in describing a log $\pmlog$ if $\Language(N) \setminus \pmlog$ is small. A model $N$
represents a \emph{generalization} of log $\pmlog$ with respect to system ${\sys}$ if some behavior in ${\sys}
\setminus \pmlog$ exists in $\Language(N)$. Finally, a model $N$ is \emph{simple} when it has the minimal
complexity in representing $\Language(N)$, i.e., the well-known \emph{Occam's razor principle}.
It is widely acknowledged that the size of a process model is the
most important simplicity indicator~\cite{AalstBook}. However, as we will see in Section~\ref{sec:ssr}, this paper introduces
a tailored simplicity metric which is sensitive to the type of models considered.

\section{Parikh Vector} 
\label{sec:2.parikh}

\section{Domínios númericos abstractos y descubrimiento de procesos} 
\label{sec:2.discovery}

\section{Información negativa} 
\label{sec:2.negative}

\section{Enunciado del problema} 
\label{sec:2.problem}

Dado un log $\pmlogp$ y un conjunto de trazas negativas $\pmlogn$ el objetivo de las técnicas utilizadas 
en el desarrollo de la tesina es derivar $N$ tal que posea las siguientes caraterísticas:\\

\begin{itemize}
 \item $N$ es una red de Petri pura ponderada;
 \item $\forall \sigmap \in \pmlogp:~ \sigmap \in \Language(N)$;
 \item $\forall \sigman \in \pmlogn:~ \sigman \notin \Language(N)$;
 \item $N$ minimiza su complejidad. 
\end{itemize}

Los primers tres items son triviales ya que son garantizados por la metodología utilizada
para generar la red.

\todo[inline]{traducir el texto que sigue.}
, while for simplicity we will evaluate derived models with respect to a tailored fine-grain simplicity metric
which takes into account not only the number of elements but also its weight. In the evaluation section, we will use current metrics for precision and
generalization to estimate whereas the derived models are in good balance between underfitting and overfitting the log.
\todo[inline]{Primero debería entenderlo....}

\section{Resumen del capítulo}
\label{sec:2.resumen}
En este capítulo se hizo una introducción a diferentes temas fundamentales para la comprensión y el desarrollo de este trabajo. 
Se presentaron conceptos cruciales como \textit{redes de Petri} y \textit{minería de procesos}, definiciones como la
de \textit{Parikh Vector} y se describireron las nociones utilizadas de \textit{logs}, \textit{información negativa} o el concepto 
entendido por complejidade de una red de Petri.
\todo[inline]{rellenar el resumen del capítulo}
En particular, 
\todo[inline]{rellenar el resumen del capítulo. Primero sería bueno terminarlo para contar que dice, no?}
Por último, utilizando estos conceptos, se presentó una definición formal del problema a tratar.
